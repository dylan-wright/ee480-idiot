%   Dylan Wright - dylan.wright@uky.edu
%   EE480 - Assignment 2: The Making Of An IDIOT
%   note.tex : Implementer's Note
%   Version:
%       02-14-2016 : initial
%	  03-06-2016 : Revised content slightly, added "General Approach" and 
%			      "Issues" subsection. Wrote a basic abstract  that may be
%				revised later. 

\documentclass[conference]{IEEEtran}
\usepackage{graphicx}

\begin{document}
\title{Assignment 2: The Making Of An IDIOT\\Implementor's Notes}
\author{\IEEEauthorblockN{Dylan Wright}
        \IEEEauthorblockA{dylan.wright@uky.edu}
        \IEEEauthorblockN{Casey O'Kane}
        \IEEEauthorblockA{casey.okane@uky.edu}}

\maketitle

\begin{abstract}
Goal of this assignment involved the implementation of the IDIOT instruction set
using the AIK assembler, the Verilog Hardware Design Language and detailed test
plan to exhaustively test the different components and logic of the design. 
\end{abstract}

\section{Testing}
\subsection{Instruction Set Architecture}
In order to test the IDIOT instruction set specification a test framework was
implemented. This framework is in the \texttt{IDIOT/} directory. The framework
consists of the folowing files:
\subsubsection{\texttt{aik.py}}
To automatically test files \texttt{aik.py} sends a PUSH request to the
AIK cgi program. The returned html page is parsed and each section is output.
The \texttt{.text} and \texttt{.data} sections are sent to stdout and the
assembler messages are sent to \texttt{stderr}. This method is not ideal, an
AIK executable would be prefferable. Sample run:
\begin{verbatim}
 $ echo "file.idiot" | ./aik.py
\end{verbatim}
\subsubsection{\texttt{diss.py}}
To make test results human readable, \texttt{diss.py} dissassembles a 
\texttt{.out} file (the \texttt{.text} and \texttt{.data} segment of the 
output of \texttt{aik.py}). The code is converted to binary and displayed in
tabular format. Sample run:
\begin{verbatim}
 $ echo "file.out" | ./diss.py
\end{verbatim}
\subsubsection{\texttt{test.sh}}
This file can be used to test each \texttt{.idiot} file in the \texttt{progs/}
directory. This script runs each file through AIK and compares the output to
the expected output. \texttt{.text} and \texttt{.data} segment expected
output should be placed in a file with the same name as the program and a 
\texttt{.expected.out} file extension. Expected assembler messages should
be placed in a file with a \texttt{.expected.err} extension. The test script
will report the number of passed, failed and possibly failed tests. This test
framework was adapted from a script provided by Dr. Jaromczyk in the Fall
2015 CS 441G: Compilers course. Sample run:
\begin{verbatim}
 $ ./test
\end{verbatim}
\subsection{Verilog Modules}
\section{General Approach}
\section{Issues}
\end{document}
